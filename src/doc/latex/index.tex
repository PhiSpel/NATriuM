\hypertarget{index_lbm_sec}{}\section{The Lattice Boltzmann Method}\label{index_lbm_sec}
Lattice Boltzmann (L\-B) methods are an alternative approach for the simulation of fluid flows. Modeling particulate processes (advection and collision) they indirectly solve the fluid mechanical Navier-\/\-Stokes equations. Since 1988 L\-B has established besides classical finite element, finite volume and finite difference schemes, particularly for the simulation of complex flows. Concretely, they provide a straightforwardmodeling of multiphase flows and flows in complex geometries. Both kinds of simulations are nowadays indispensable, especially in engineering applications. The L\-B method has proven suitable for simulating complex flows on parallel computers. However, the original L\-B algorithm is limited to regular computation grids (the lattices), which is a severe shortcoming, specifically in resolving different scales and flows at high Reynolds numbers (Re). Furthermore, it is restricted to at most second order accuracy in time and space due to the inherent Euler time integration. Consequently, researchers attempted to transfer the L\-B concept to irregular grids and higher order accuracy. Various approaches have been published.\hypertarget{index_sedg_sec}{}\section{L\-B\-M on unstructured grids}\label{index_sedg_sec}
A very promising approach for L\-B\-M on irregular grids was published by Lee and Lin in 2003. They managed to split the Boltzmann equation into collision and advection, without restricting it to regular grids. They could solve the advection equation (streaming step) using a Finite Element scheme on a triangular grid with high order and even implicit time integrators. Their method was picked up by Min and Lee in 2011, who used a spectral element discontinuous Galerkin (S\-E\-D\-G) solver for the advection and obtained a high-\/order convergent S\-E\-D\-G-\/\-L\-B\-M scheme. N\-A\-Triu\-M is based on this approach. Obviously, the sophisticated streaming procedure comes at the price of higher computation time. We have measured factors of $\sim$60 in a comparison with the Open Source L\-B\-M software Palabos (for B\-G\-K-\/collision and equal C\-F\-L numbers).\hypertarget{index_tpl_sec}{}\section{Third party libraries (\-T\-P\-Ls)}\label{index_tpl_sec}
N\-A\-Triu\-M inherits a huge part of its functionality from third party libraries, most importantly deal.\-I\-I. Deal.\-I\-I is a finite element library which was awarded the Wilkinson price for numerical software in 2007. Together with a huge F\-E\-M functionality, it offers wrappers to other T\-P\-Ls like Trilinos to speed up simulations on multicore architectures. We use the Trilinos data structures (sparse matrices and vectors). The boost unit test module is used to structure the testing code. C\-Make is used for user-\/friendly compilation.\hypertarget{index_struct_sec}{}\section{Code structure}\label{index_struct_sec}
At the moment, N\-A\-Triu\-M can only be used as a library, but it is going to be usable from the command line in a later stage of the project. A class diagram is in the documentation folder, but it could be slightly outdated. The source directory contains the following folders\-:
\begin{DoxyItemize}
\item natrium\-: N\-A\-Triu\-M library. Contains the code for the S\-E\-D\-G-\/\-Solver.
\item test\-: Unit and integration tests.
\item doc\-: Code documentation. Most importantly, this Doxygen documentation.
\item examples\-: Example simulations that use the N\-A\-Triu\-M code.
\item analysis\-: Scripts for numerical analysis of the code (convergence tests, etc.)
\item preprocessing\-: Parameter G\-U\-I.
\item postprocessing\-: Nothing special (at the moment).
\end{DoxyItemize}\hypertarget{index_workflow_sec}{}\section{Workflow}\label{index_workflow_sec}
It is recommended to use the code in connection with sophisticated pre-\/ and postprocessing tools.
\begin{DoxyItemize}
\item Preprocessing\-: Deal.\-I\-I supports various mesh grid formats. E.\-g. Salome can be used for grid generation.
\item Postprocessing\-: Paraview is highly recommended.
\end{DoxyItemize}

The Open Fuel Cell simulator (Open\-F\-C\-S\-T) also uses the workflow Salome-\/$>$deal.\-I\-I-\/$>$Paraview. It might be interesting for you to take a look at the Open\-F\-C\-S\-T user manual (online available) which includes for example a short Tutorial on interfacing Salome and deal.\-I\-I.\hypertarget{index_start_sec}{}\section{Getting started}\label{index_start_sec}
For simple applications, the triangulation can be generated within deal.\-I\-I. The examples in the Examples section will help you to get grip on the code.\hypertarget{index_install_sec}{}\section{Installation}\label{index_install_sec}
The installation of N\-A\-T\-Riu\-M requires several third party libraries. For a detailed documentation see the file I\-N\-S\-T\-A\-L\-L.\-txt. 